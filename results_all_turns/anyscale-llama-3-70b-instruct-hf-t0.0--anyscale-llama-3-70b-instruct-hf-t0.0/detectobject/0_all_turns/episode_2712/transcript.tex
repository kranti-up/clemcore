
\documentclass{article}
\usepackage{colortbl}
\usepackage{makecell}
\usepackage{multirow}
\usepackage{supertabular}

\begin{document}

\newcounter{utterance}

\twocolumn

{ \footnotesize  \setcounter{utterance}{1}
\setlength{\tabcolsep}{0pt}
\begin{supertabular}{c@{$\;$}|p{.15\linewidth}@{}p{.15\linewidth}p{.15\linewidth}p{.15\linewidth}p{.15\linewidth}p{.15\linewidth}}

    \# & $\;$A & \multicolumn{4}{c}{Game Master} & $\;\:$B\\
    \hline 

    \theutterance \stepcounter{utterance}  

    & & \multicolumn{4}{p{0.6\linewidth}}{\cellcolor[rgb]{0.9,0.9,0.9}{%
	\makecell[{{p{\linewidth}}}]{% 
	  \tt {\tiny [A$\langle$GM]}  
	 You are a helpful assistant tasked with identifying specific object IDs from references made in conversation snippets.\\ \tt \\ \tt The scene contains multiple objects, each identified by an ID and characterized by attributes including name, color, and size. Analyze the current utterance to determine which objects are being mentioned based on their described attributes. Match these details with the attributes of objects in the scene to accurately identify and return the corresponding object IDs. If an utterance refers to multiple objects, detect and respond with all relevant object IDs. To understand the context better, the dialogue history, if available, is shared under the label 'Dialogue History'.\\ \tt \\ \tt Dilaogue History:\\ \tt User: Do you have any coats with a plain pattern?\\ \tt System: We have several coats with a plain pattern, including a black coat on the right side of the far rack on the right, a brown coat on the middle right in the middle group of cubbyholes against the far wall, and a black coat on the left side of the rack on the left. What do you think?\\ \tt User: I'm looking to buy something with a pattern and size like the black coat or the brown coat.\\ \tt System: Similar items include a black coat on the left side of the rack on the left, a grey coat on the right side of the far rack on the right, and a brown coat fifth from right on the middle row of cubbyholes against the far wall.\\ \tt User: What sizes are the brown coat and the black coat?\\ \tt System: The brown coat is size XL and the black coat is size L.\\ \tt User: What else do you have?\\ \tt System: We have a grey coat to the right of center on the near rack, a black coat to the right of center on the far rack on the right, and a black coat in the cubbyhole on the left side of the middle row, in the group of cubbyholes on the right against the far wall.\\ \tt \\ \tt \\ \tt Below is the scene description. Each line contains a tuple with the format (object_id, type, color, size), describing each object in the scene.\\ \tt \\ \tt object_id, object_type, object_color, object_size\\ \tt 94, coat, black, L\\ \tt 25, coat, brown, XL\\ \tt 121, coat, grey, XXL\\ \tt \\ \tt \\ \tt Your response should list the matched object IDs under the label 'ObjectID', followed by a newline.\\ \tt \\ \tt Do not generate any other text or explanations.\\ \tt \\ \tt Let's begin\\ \tt User: Can you tell me about the reviews of the black coat and the gray coat? 
	  } 
	   } 
	   } 
	 & \\ 
 

    \theutterance \stepcounter{utterance}  

    & \multicolumn{4}{p{0.6\linewidth}}{\cellcolor[rgb]{0.8,1,0.9}{%
	\makecell[{{p{\linewidth}}}]{% 
	  \tt {\tiny [A$\rangle$GM]}  
	 ObjectID: 94, 121 
	  } 
	   } 
	   } 
	 & & \\ 
 

    \theutterance \stepcounter{utterance}  

    & & & \multicolumn{2}{p{0.3\linewidth}}{\cellcolor[rgb]{0.95,0.95,0.95}{%
	\makecell[{{p{\linewidth}}}]{% 
	  \tt {\tiny [GM$|$GM]}  
	 {'prediction': [94, 121], 'groundtruth': [94, 121]} 
	  } 
	   } 
	   } 
	 & & \\ 
 

    \theutterance \stepcounter{utterance}  

    & & & \multicolumn{2}{p{0.3\linewidth}}{\cellcolor[rgb]{0.95,0.95,0.95}{%
	\makecell[{{p{\linewidth}}}]{% 
	  \tt {\tiny [GM$|$GM]}  
	 dialogue completed 
	  } 
	   } 
	   } 
	 & & \\ 
 

    \theutterance \stepcounter{utterance}  

    & & & \multicolumn{2}{p{0.3\linewidth}}{\cellcolor[rgb]{0.95,0.95,0.95}{%
	\makecell[{{p{\linewidth}}}]{% 
	  \tt {\tiny [GM$|$GM]}  
	 end game 
	  } 
	   } 
	   } 
	 & & \\ 
 

\end{supertabular}
}

\end{document}
