
\documentclass{article}
\usepackage{colortbl}
\usepackage{makecell}
\usepackage{multirow}
\usepackage{supertabular}

\begin{document}

\newcounter{utterance}

\twocolumn

{ \footnotesize  \setcounter{utterance}{1}
\setlength{\tabcolsep}{0pt}
\begin{supertabular}{c@{$\;$}|p{.15\linewidth}@{}p{.15\linewidth}p{.15\linewidth}p{.15\linewidth}p{.15\linewidth}p{.15\linewidth}}

    \# & $\;$A & \multicolumn{4}{c}{Game Master} & $\;\:$B\\
    \hline 

    \theutterance \stepcounter{utterance}  

    & & \multicolumn{4}{p{0.6\linewidth}}{\cellcolor[rgb]{0.9,0.9,0.9}{%
	\makecell[{{p{\linewidth}}}]{% 
	  \tt {\tiny [A$\langle$GM]}  
	 You are a helpful assistant tasked with identifying specific object IDs from references made in conversation snippets.\\ \tt \\ \tt The scene contains multiple objects, each identified by an ID and characterized by attributes including name, color, and size. Analyze the current utterance to determine which objects are being mentioned based on their described attributes. Match these details with the attributes of objects in the scene to accurately identify and return the corresponding object IDs. If an utterance refers to multiple objects, detect and respond with all relevant object IDs. To understand the context better, the dialogue history, if available, is shared under the label 'Dialogue History'.\\ \tt \\ \tt Dilaogue History:\\ \tt User: Which of these tops would you suggest I take a look at?\\ \tt System: I'd suggest the maroon Christmas sweater closest to you in the top row, the gray and white one below it, or the gray and white one in the middle of the bottom row.\\ \tt User: Can you tell me about both of the gray and white tops?\\ \tt System: They have the same general info.  Both are L, come in XXL, L, S, made by North Lodge, short sleeve, light spot pattern, and have a 2.7 rating and cost 54.99.\\ \tt User: Can you repeat the price for that?  And tell me about the Christmas sweater?\\ \tt System: For which gray top?\\ \tt User: The one closest to me, and also the price of the Christmas one.\\ \tt System: The gray one is 54.99, the Christmas one is 9.99.\\ \tt User: I'd like to see if you have anything in pink?\\ \tt System: No, not at the moment.  Could I help you find something else?\\ \tt User: Oh, for now, I'll take the Christmas sweater, and that gray and white one under it.\\ \tt System: Great, I'll add the sweater and the top to your cart for you now.\\ \tt User: Have you got any sweaters from The Vegan Baker?\\ \tt System: Closest to you in the middle tier, I have that purple sweater by them.\\ \tt \\ \tt \\ \tt Below is the scene description. Each line contains a tuple with the format (object_id, type, color, size), describing each object in the scene.\\ \tt \\ \tt object_id, object_type, object_color, object_size\\ \tt 14, blouse, grey, white, L\\ \tt 13, blouse, grey, white, L\\ \tt 9, blouse, maroon, white, blue, XS\\ \tt \\ \tt \\ \tt Your response should list the matched object IDs under the label 'ObjectID', followed by a newline.\\ \tt \\ \tt Do not generate any other text or explanations.\\ \tt \\ \tt Let's begin\\ \tt User: Could you tell me about the two shirts in my cart? 
	  } 
	   } 
	   } 
	 & \\ 
 

    \theutterance \stepcounter{utterance}  

    & \multicolumn{4}{p{0.6\linewidth}}{\cellcolor[rgb]{0.8,1,0.9}{%
	\makecell[{{p{\linewidth}}}]{% 
	  \tt {\tiny [A$\rangle$GM]}  
	 ObjectID: 9, 13 
	  } 
	   } 
	   } 
	 & & \\ 
 

    \theutterance \stepcounter{utterance}  

    & & & \multicolumn{2}{p{0.3\linewidth}}{\cellcolor[rgb]{0.95,0.95,0.95}{%
	\makecell[{{p{\linewidth}}}]{% 
	  \tt {\tiny [GM$|$GM]}  
	 {'prediction': [9, 13], 'groundtruth': [14, 9]} 
	  } 
	   } 
	   } 
	 & & \\ 
 

    \theutterance \stepcounter{utterance}  

    & & & \multicolumn{2}{p{0.3\linewidth}}{\cellcolor[rgb]{0.95,0.95,0.95}{%
	\makecell[{{p{\linewidth}}}]{% 
	  \tt {\tiny [GM$|$GM]}  
	 dialogue completed 
	  } 
	   } 
	   } 
	 & & \\ 
 

    \theutterance \stepcounter{utterance}  

    & & & \multicolumn{2}{p{0.3\linewidth}}{\cellcolor[rgb]{0.95,0.95,0.95}{%
	\makecell[{{p{\linewidth}}}]{% 
	  \tt {\tiny [GM$|$GM]}  
	 end game 
	  } 
	   } 
	   } 
	 & & \\ 
 

\end{supertabular}
}

\end{document}
