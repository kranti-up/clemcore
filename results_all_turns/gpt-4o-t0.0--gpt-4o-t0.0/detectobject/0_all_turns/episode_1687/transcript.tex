
\documentclass{article}
\usepackage{colortbl}
\usepackage{makecell}
\usepackage{multirow}
\usepackage{supertabular}

\begin{document}

\newcounter{utterance}

\twocolumn

{ \footnotesize  \setcounter{utterance}{1}
\setlength{\tabcolsep}{0pt}
\begin{supertabular}{c@{$\;$}|p{.15\linewidth}@{}p{.15\linewidth}p{.15\linewidth}p{.15\linewidth}p{.15\linewidth}p{.15\linewidth}}

    \# & $\;$A & \multicolumn{4}{c}{Game Master} & $\;\:$B\\
    \hline 

    \theutterance \stepcounter{utterance}  

    & & \multicolumn{4}{p{0.6\linewidth}}{\cellcolor[rgb]{0.9,0.9,0.9}{%
	\makecell[{{p{\linewidth}}}]{% 
	  \tt {\tiny [A$\langle$GM]}  
	 You are a helpful assistant tasked with identifying specific object IDs from references made in conversation snippets.\\ \tt \\ \tt The scene contains multiple objects, each identified by an ID and characterized by attributes including name, color, and size. Analyze the current utterance to determine which objects are being mentioned based on their described attributes. Match these details with the attributes of objects in the scene to accurately identify and return the corresponding object IDs. If an utterance refers to multiple objects, detect and respond with all relevant object IDs. To understand the context better, the dialogue history, if available, is shared under the label 'Dialogue History'.\\ \tt \\ \tt Dilaogue History:\\ \tt User: Hello, I need a new shirt to match the rest of my clothes.\\ \tt System: There are several on the table, including violet, grey and brown options.\\ \tt User: Do you have anything similar to the violet or grey shirts but with short sleeves?\\ \tt System: Sorry, I don't seem to have anything like that right now.\\ \tt User: What do you have with good reviews and short sleeves?\\ \tt System: I don't have any shirts matching that description right now.\\ \tt User: Are there any nice sweaters?\\ \tt System: There are, on the same table as the shirts you can see a dark violet sweater, a grey sweater and a dark grey sweater.\\ \tt User: Ok.  I think I'm going to take the grey shirt, it's in the middle of the second tier of the table.\\ \tt System: Alright, it's been added to your cart.\\ \tt User: Are there any good sweaters here?\\ \tt System: There's a green sweater at the back right of the table, as well as a couple of grey ones on the second tier.\\ \tt User: Is there anything else with good reviews and an affordable price tag?\\ \tt System: The grey sweater on the bottom shelf of the table fits that desctiption.\\ \tt \\ \tt \\ \tt Below is the scene description. Each line contains a tuple with the format (object_id, type, color, size), describing each object in the scene.\\ \tt \\ \tt object_id, object_type, object_color, object_size\\ \tt 75, shirt, violet, M\\ \tt 73, shirt, grey, XS\\ \tt 78, blouse, black, XL\\ \tt 55, blouse, maroon, white, blue, S\\ \tt \\ \tt \\ \tt Your response should list the matched object IDs under the label 'ObjectID', followed by a newline.\\ \tt \\ \tt Do not generate any other text or explanations.\\ \tt \\ \tt Let's begin\\ \tt User: Ok.  I think I'm actually going to take the violet shirt. 
	  } 
	   } 
	   } 
	 & \\ 
 

    \theutterance \stepcounter{utterance}  

    & \multicolumn{4}{p{0.6\linewidth}}{\cellcolor[rgb]{0.8,1,0.9}{%
	\makecell[{{p{\linewidth}}}]{% 
	  \tt {\tiny [A$\rangle$GM]}  
	 ObjectID\\ \tt 75 
	  } 
	   } 
	   } 
	 & & \\ 
 

    \theutterance \stepcounter{utterance}  

    & & & \multicolumn{2}{p{0.3\linewidth}}{\cellcolor[rgb]{0.95,0.95,0.95}{%
	\makecell[{{p{\linewidth}}}]{% 
	  \tt {\tiny [GM$|$GM]}  
	 {'prediction': [75], 'groundtruth': [75]} 
	  } 
	   } 
	   } 
	 & & \\ 
 

    \theutterance \stepcounter{utterance}  

    & & & \multicolumn{2}{p{0.3\linewidth}}{\cellcolor[rgb]{0.95,0.95,0.95}{%
	\makecell[{{p{\linewidth}}}]{% 
	  \tt {\tiny [GM$|$GM]}  
	 dialogue completed 
	  } 
	   } 
	   } 
	 & & \\ 
 

    \theutterance \stepcounter{utterance}  

    & & & \multicolumn{2}{p{0.3\linewidth}}{\cellcolor[rgb]{0.95,0.95,0.95}{%
	\makecell[{{p{\linewidth}}}]{% 
	  \tt {\tiny [GM$|$GM]}  
	 end game 
	  } 
	   } 
	   } 
	 & & \\ 
 

\end{supertabular}
}

\end{document}
