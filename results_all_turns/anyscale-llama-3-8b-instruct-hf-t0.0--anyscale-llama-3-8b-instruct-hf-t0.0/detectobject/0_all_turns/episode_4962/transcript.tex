
\documentclass{article}
\usepackage{colortbl}
\usepackage{makecell}
\usepackage{multirow}
\usepackage{supertabular}

\begin{document}

\newcounter{utterance}

\twocolumn

{ \footnotesize  \setcounter{utterance}{1}
\setlength{\tabcolsep}{0pt}
\begin{supertabular}{c@{$\;$}|p{.15\linewidth}@{}p{.15\linewidth}p{.15\linewidth}p{.15\linewidth}p{.15\linewidth}p{.15\linewidth}}

    \# & $\;$A & \multicolumn{4}{c}{Game Master} & $\;\:$B\\
    \hline 

    \theutterance \stepcounter{utterance}  

    & & \multicolumn{4}{p{0.6\linewidth}}{\cellcolor[rgb]{0.9,0.9,0.9}{%
	\makecell[{{p{\linewidth}}}]{% 
	  \tt {\tiny [A$\langle$GM]}  
	 You are a helpful assistant tasked with identifying specific object IDs from references made in conversation snippets.\\ \tt \\ \tt The scene contains multiple objects, each identified by an ID and characterized by attributes including name, color, and size. Analyze the current utterance to determine which objects are being mentioned based on their described attributes. Match these details with the attributes of objects in the scene to accurately identify and return the corresponding object IDs. If an utterance refers to multiple objects, detect and respond with all relevant object IDs. To understand the context better, the dialogue history, if available, is shared under the label 'Dialogue History'.\\ \tt \\ \tt Dilaogue History:\\ \tt User: Can you show me some blouses with light spots?\\ \tt System: Of course! I have three grey and white blouses with light spots, what do you think of them? There's one in the left-most cubby, one in the next cubby to the right, and another in the third cubby from the far end.\\ \tt User: Oh those aren't bad! Do you have anything like that second one, or the white and black shirt next to it, but with long sleeves?\\ \tt System: Yes, there's a black blouse towards the far end near the last grey and white blouse that might interest you.\\ \tt User: Ok, I'll think about that for now. In the meantime, could you show me some plain jackets in size small?\\ \tt System: Yes, I can! I have two plain jackets in size S, a brown jacket on the very left and a light grey one on the other side of the same cubby. What do you think of those?\\ \tt User: I don't know, but I do love those pants! Please add both the light blue jeans and the dark green joggers to my cart.\\ \tt System: Of course, I've added both items for you.\\ \tt \\ \tt \\ \tt Below is the scene description. Each line contains a tuple with the format (object_id, type, color, size), describing each object in the scene.\\ \tt \\ \tt object_id, object_type, object_color, object_size\\ \tt 1, blouse, white, black, S\\ \tt 9, blouse, white, black, S\\ \tt 12, joggers, dark green, XL\\ \tt 13, jeans, light blue, L\\ \tt 6, jacket, brown, S\\ \tt 8, jacket, light grey, XL\\ \tt \\ \tt \\ \tt Your response should list the matched object IDs under the label 'ObjectID', followed by a newline.\\ \tt \\ \tt Do not generate any other text or explanations.\\ \tt \\ \tt Let's begin\\ \tt User: Alright. Do you have anything similar to those two jackets, but in brown and from The Vegan Baker? 
	  } 
	   } 
	   } 
	 & \\ 
 

    \theutterance \stepcounter{utterance}  

    & \multicolumn{4}{p{0.6\linewidth}}{\cellcolor[rgb]{0.8,1,0.9}{%
	\makecell[{{p{\linewidth}}}]{% 
	  \tt {\tiny [A$\rangle$GM]}  
	 ObjectID\\ \tt 6 
	  } 
	   } 
	   } 
	 & & \\ 
 

    \theutterance \stepcounter{utterance}  

    & & & \multicolumn{2}{p{0.3\linewidth}}{\cellcolor[rgb]{0.95,0.95,0.95}{%
	\makecell[{{p{\linewidth}}}]{% 
	  \tt {\tiny [GM$|$GM]}  
	 {'prediction': [6], 'groundtruth': [6, 8]} 
	  } 
	   } 
	   } 
	 & & \\ 
 

    \theutterance \stepcounter{utterance}  

    & & & \multicolumn{2}{p{0.3\linewidth}}{\cellcolor[rgb]{0.95,0.95,0.95}{%
	\makecell[{{p{\linewidth}}}]{% 
	  \tt {\tiny [GM$|$GM]}  
	 dialogue completed 
	  } 
	   } 
	   } 
	 & & \\ 
 

    \theutterance \stepcounter{utterance}  

    & & & \multicolumn{2}{p{0.3\linewidth}}{\cellcolor[rgb]{0.95,0.95,0.95}{%
	\makecell[{{p{\linewidth}}}]{% 
	  \tt {\tiny [GM$|$GM]}  
	 end game 
	  } 
	   } 
	   } 
	 & & \\ 
 

\end{supertabular}
}

\end{document}
