
\documentclass{article}
\usepackage{colortbl}
\usepackage{makecell}
\usepackage{multirow}
\usepackage{supertabular}

\begin{document}

\newcounter{utterance}

\twocolumn

{ \footnotesize  \setcounter{utterance}{1}
\setlength{\tabcolsep}{0pt}
\begin{supertabular}{c@{$\;$}|p{.15\linewidth}@{}p{.15\linewidth}p{.15\linewidth}p{.15\linewidth}p{.15\linewidth}p{.15\linewidth}}

    \# & $\;$A & \multicolumn{4}{c}{Game Master} & $\;\:$B\\
    \hline 

    \theutterance \stepcounter{utterance}  

    & & \multicolumn{4}{p{0.6\linewidth}}{\cellcolor[rgb]{0.9,0.9,0.9}{%
	\makecell[{{p{\linewidth}}}]{% 
	  \tt {\tiny [A$\langle$GM]}  
	 You are a helpful assistant tasked with identifying specific object IDs from references made in conversation snippets.\\ \tt \\ \tt The scene contains multiple objects, each identified by an ID and characterized by attributes including name, color, and size. Analyze the current utterance to determine which objects are being mentioned based on their described attributes. Match these details with the attributes of objects in the scene to accurately identify and return the corresponding object IDs. If an utterance refers to multiple objects, detect and respond with all relevant object IDs. To understand the context better, the dialogue history, if available, is shared under the label 'Dialogue History'.\\ \tt \\ \tt Dilaogue History:\\ \tt User: Can you show me jackets?\\ \tt System: Do you like the light grey one hanging up on the top of the wall? It is a jacket.\\ \tt User: I'd like to see pants with good ratings.\\ \tt System: Do you like the grey pair in the cubby to the right? They have a good rating.\\ \tt User: Can you show me options in grey with heavy stripes?\\ \tt System: I'm sorry, I don't have any pants like that right now.\\ \tt User: Can you put the purple dress from the right side of the wall in my cart along with the grey jacket up above it?\\ \tt System: Yes, I will put those items in your cart.\\ \tt User: Can you show me some pricier sweaters?\\ \tt System: Do you like the black one hanging up on the top left of the wall? It is a pricier sweater.\\ \tt User: How about joggers by Downtown Consignment?\\ \tt System: I have the dark green pair in the second cubby from the back.\\ \tt User: Any others?\\ \tt System: That's all I have for joggers by Downtown Consignment right now.\\ \tt User: Can you show me pants with a higher price point?\\ \tt System: I can help with that. I have the green pair in the very back cubby and the grey pair toward the front.\\ \tt \\ \tt \\ \tt Below is the scene description. Each line contains a tuple with the format (object_id, type, color, size), describing each object in the scene.\\ \tt \\ \tt object_id, object_type, object_color, object_size\\ \tt 44, dress, purple, XL\\ \tt 39, jacket, light grey, XL\\ \tt 95, trousers, green, XL\\ \tt 83, trousers, grey, XS\\ \tt \\ \tt \\ \tt Your response should list the matched object IDs under the label 'ObjectID', followed by a newline.\\ \tt \\ \tt Do not generate any other text or explanations.\\ \tt \\ \tt Let's begin\\ \tt User: I'll buy both of those. 
	  } 
	   } 
	   } 
	 & \\ 
 

    \theutterance \stepcounter{utterance}  

    & \multicolumn{4}{p{0.6\linewidth}}{\cellcolor[rgb]{0.8,1,0.9}{%
	\makecell[{{p{\linewidth}}}]{% 
	  \tt {\tiny [A$\rangle$GM]}  
	 ObjectID\\ \tt 44\\ \tt 39 
	  } 
	   } 
	   } 
	 & & \\ 
 

    \theutterance \stepcounter{utterance}  

    & & & \multicolumn{2}{p{0.3\linewidth}}{\cellcolor[rgb]{0.95,0.95,0.95}{%
	\makecell[{{p{\linewidth}}}]{% 
	  \tt {\tiny [GM$|$GM]}  
	 {'prediction': [44, 39], 'groundtruth': [95, 83]} 
	  } 
	   } 
	   } 
	 & & \\ 
 

    \theutterance \stepcounter{utterance}  

    & & & \multicolumn{2}{p{0.3\linewidth}}{\cellcolor[rgb]{0.95,0.95,0.95}{%
	\makecell[{{p{\linewidth}}}]{% 
	  \tt {\tiny [GM$|$GM]}  
	 dialogue completed 
	  } 
	   } 
	   } 
	 & & \\ 
 

    \theutterance \stepcounter{utterance}  

    & & & \multicolumn{2}{p{0.3\linewidth}}{\cellcolor[rgb]{0.95,0.95,0.95}{%
	\makecell[{{p{\linewidth}}}]{% 
	  \tt {\tiny [GM$|$GM]}  
	 end game 
	  } 
	   } 
	   } 
	 & & \\ 
 

\end{supertabular}
}

\end{document}
